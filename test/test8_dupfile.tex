
%
% NOTE: THIS FILE WAS AUTOMATICALLY GENERATED BY bibolamazi SCRIPT!
%       ANY CHANGES WILL BE LOST!
%
% File automatically generated by bibolamazi's `duplicates` filter.
%
% You should include this file in your main LaTeX file with the command
%
%   
%
% NOTE: THIS FILE WAS AUTOMATICALLY GENERATED BY bibolamazi SCRIPT!
%       ANY CHANGES WILL BE LOST!
%
% File automatically generated by bibolamazi's `duplicates` filter.
%
% You should include this file in your main LaTeX file with the command
%
%   
%
% NOTE: THIS FILE WAS AUTOMATICALLY GENERATED BY bibolamazi SCRIPT!
%       ANY CHANGES WILL BE LOST!
%
% File automatically generated by bibolamazi's `duplicates` filter.
%
% You should include this file in your main LaTeX file with the command
%
%   
%
% NOTE: THIS FILE WAS AUTOMATICALLY GENERATED BY bibolamazi SCRIPT!
%       ANY CHANGES WILL BE LOST!
%
% File automatically generated by bibolamazi's `duplicates` filter.
%
% You should include this file in your main LaTeX file with the command
%
%   \input{test8_dupfile.tex}
%
% in your document preamble.
%



%
% The following will define the command \bibalias{<alias>}{<source>}, which will make
% the command \cite[..]{<alias>} the same as doing \cite[..]{<source>}.
%
% This code has been copied and adapted from
%    http://tex.stackexchange.com/questions/37233/
%

\makeatletter
% \bibalias{<alias>}{<source>} makes \cite{<alias>} equivalent to \cite{<source>}
\newcommand\bibalias[2]{%
  \@namedef{bibali@#1}{#2}%
}


%
% Note: The `\cite` command provided here does not accept spaces in/between its
% arguments. This might be tricky, since revTeX does accept those spaces. You
% can work around by using LaTeX comments which automatically remove the
% following space after newline, in the following way:
%
%    \cite{key1,%
%          key2,%
%          key3%
%    }
%
% Make sure you don't add space between the comma and the percent sign.
%

\newtoks\biba@toks
\let\bibalias@oldcite\cite
\def\cite{%
  \@ifnextchar[{%
    \biba@cite@optarg%
  }{%
    \biba@cite{}%
  }%
}
\newcommand\biba@cite@optarg[2][]{%
  \biba@cite{[#1]}{#2}%
}
\newcommand\biba@cite[2]{%
  \biba@toks{\bibalias@oldcite#1}%
  \def\biba@comma{}%
  \def\biba@all{}%
  \@for\biba@one:=#2\do{%
    \@ifundefined{bibali@\biba@one}{%
      \edef\biba@all{\biba@all\biba@comma\biba@one}%
    }{%
      \PackageInfo{bibalias}{%
        Replacing citation `\biba@one' with `\@nameuse{bibali@\biba@one}'
      }%
      \edef\biba@all{\biba@all\biba@comma\@nameuse{bibali@\biba@one}}%
    }%
    \def\biba@comma{,}%
  }%
  %
  % However, still write in the .aux file a dummy \citation{...} command, so that
  % filters.util.auxfile will still catch those used citations....
  %
  %\immediate\write\@auxout{\noexpand{\bgroup\newcommand\citation[2][]{}\citation{#2}\egroup}}
  \immediate\write\@auxout{\noexpand\bgroup\noexpand\renewcommand\noexpand\citation[1]{}\noexpand\citation{#2}\noexpand\egroup}%
  %
  % Now, produce the \cite command with the original keys instead of the aliases
  %
  \edef\biba@tmp{\the\biba@toks{\biba@all}}%
  \biba@tmp%
}
\makeatother


%
% Now, declare all the alias keys.
%

\bibalias{doi:10.1103/PhysRev.47.777}{doi:10.1103/PhysRev.47.777--EPR-paper}



%
% in your document preamble.
%



%
% The following will define the command \bibalias{<alias>}{<source>}, which will make
% the command \cite[..]{<alias>} the same as doing \cite[..]{<source>}.
%
% This code has been copied and adapted from
%    http://tex.stackexchange.com/questions/37233/
%

\makeatletter
% \bibalias{<alias>}{<source>} makes \cite{<alias>} equivalent to \cite{<source>}
\newcommand\bibalias[2]{%
  \@namedef{bibali@#1}{#2}%
}


%
% Note: The `\cite` command provided here does not accept spaces in/between its
% arguments. This might be tricky, since revTeX does accept those spaces. You
% can work around by using LaTeX comments which automatically remove the
% following space after newline, in the following way:
%
%    \cite{key1,%
%          key2,%
%          key3%
%    }
%
% Make sure you don't add space between the comma and the percent sign.
%

\newtoks\biba@toks
\let\bibalias@oldcite\cite
\def\cite{%
  \@ifnextchar[{%
    \biba@cite@optarg%
  }{%
    \biba@cite{}%
  }%
}
\newcommand\biba@cite@optarg[2][]{%
  \biba@cite{[#1]}{#2}%
}
\newcommand\biba@cite[2]{%
  \biba@toks{\bibalias@oldcite#1}%
  \def\biba@comma{}%
  \def\biba@all{}%
  \@for\biba@one:=#2\do{%
    \@ifundefined{bibali@\biba@one}{%
      \edef\biba@all{\biba@all\biba@comma\biba@one}%
    }{%
      \PackageInfo{bibalias}{%
        Replacing citation `\biba@one' with `\@nameuse{bibali@\biba@one}'
      }%
      \edef\biba@all{\biba@all\biba@comma\@nameuse{bibali@\biba@one}}%
    }%
    \def\biba@comma{,}%
  }%
  %
  % However, still write in the .aux file a dummy \citation{...} command, so that
  % filters.util.auxfile will still catch those used citations....
  %
  %\immediate\write\@auxout{\noexpand{\bgroup\newcommand\citation[2][]{}\citation{#2}\egroup}}
  \immediate\write\@auxout{\noexpand\bgroup\noexpand\renewcommand\noexpand\citation[1]{}\noexpand\citation{#2}\noexpand\egroup}%
  %
  % Now, produce the \cite command with the original keys instead of the aliases
  %
  \edef\biba@tmp{\the\biba@toks{\biba@all}}%
  \biba@tmp%
}
\makeatother


%
% Now, declare all the alias keys.
%

\bibalias{doi:10.1103/PhysRev.47.777}{doi:10.1103/PhysRev.47.777--EPR-paper}



%
% in your document preamble.
%



%
% The following will define the command \bibalias{<alias>}{<source>}, which will make
% the command \cite[..]{<alias>} the same as doing \cite[..]{<source>}.
%
% This code has been copied and adapted from
%    http://tex.stackexchange.com/questions/37233/
%

\makeatletter
% \bibalias{<alias>}{<source>} makes \cite{<alias>} equivalent to \cite{<source>}
\newcommand\bibalias[2]{%
  \@namedef{bibali@#1}{#2}%
}


%
% Note: The `\cite` command provided here does not accept spaces in/between its
% arguments. This might be tricky, since revTeX does accept those spaces. You
% can work around by using LaTeX comments which automatically remove the
% following space after newline, in the following way:
%
%    \cite{key1,%
%          key2,%
%          key3%
%    }
%
% Make sure you don't add space between the comma and the percent sign.
%

\newtoks\biba@toks
\let\bibalias@oldcite\cite
\def\cite{%
  \@ifnextchar[{%
    \biba@cite@optarg%
  }{%
    \biba@cite{}%
  }%
}
\newcommand\biba@cite@optarg[2][]{%
  \biba@cite{[#1]}{#2}%
}
\newcommand\biba@cite[2]{%
  \biba@toks{\bibalias@oldcite#1}%
  \def\biba@comma{}%
  \def\biba@all{}%
  \@for\biba@one:=#2\do{%
    \@ifundefined{bibali@\biba@one}{%
      \edef\biba@all{\biba@all\biba@comma\biba@one}%
    }{%
      \PackageInfo{bibalias}{%
        Replacing citation `\biba@one' with `\@nameuse{bibali@\biba@one}'
      }%
      \edef\biba@all{\biba@all\biba@comma\@nameuse{bibali@\biba@one}}%
    }%
    \def\biba@comma{,}%
  }%
  %
  % However, still write in the .aux file a dummy \citation{...} command, so that
  % filters.util.auxfile will still catch those used citations....
  %
  %\immediate\write\@auxout{\noexpand{\bgroup\newcommand\citation[2][]{}\citation{#2}\egroup}}
  \immediate\write\@auxout{\noexpand\bgroup\noexpand\renewcommand\noexpand\citation[1]{}\noexpand\citation{#2}\noexpand\egroup}%
  %
  % Now, produce the \cite command with the original keys instead of the aliases
  %
  \edef\biba@tmp{\the\biba@toks{\biba@all}}%
  \biba@tmp%
}
\makeatother


%
% Now, declare all the alias keys.
%

\bibalias{doi:10.1103/PhysRev.47.777}{doi:10.1103/PhysRev.47.777--EPR-paper}



%
% in your document preamble.
%



%
% The following will define the command \bibalias{<alias>}{<source>}, which will make
% the command \cite[..]{<alias>} the same as doing \cite[..]{<source>}.
%
% This code has been copied and adapted from
%    http://tex.stackexchange.com/questions/37233/
%

\makeatletter
% \bibalias{<alias>}{<source>} makes \cite{<alias>} equivalent to \cite{<source>}
\newcommand\bibalias[2]{%
  \@namedef{bibali@#1}{#2}%
}


%
% Note: The `\cite` command provided here does not accept spaces in/between its
% arguments. This might be tricky, since revTeX does accept those spaces. You
% can work around by using LaTeX comments which automatically remove the
% following space after newline, in the following way:
%
%    \cite{key1,%
%          key2,%
%          key3%
%    }
%
% Make sure you don't add space between the comma and the percent sign.
%

\newtoks\biba@toks
\let\bibalias@oldcite\cite
\def\cite{%
  \@ifnextchar[{%
    \biba@cite@optarg%
  }{%
    \biba@cite{}%
  }%
}
\newcommand\biba@cite@optarg[2][]{%
  \biba@cite{[#1]}{#2}%
}
\newcommand\biba@cite[2]{%
  \biba@toks{\bibalias@oldcite#1}%
  \def\biba@comma{}%
  \def\biba@all{}%
  \@for\biba@one:=#2\do{%
    \@ifundefined{bibali@\biba@one}{%
      \edef\biba@all{\biba@all\biba@comma\biba@one}%
    }{%
      \PackageInfo{bibalias}{%
        Replacing citation `\biba@one' with `\@nameuse{bibali@\biba@one}'
      }%
      \edef\biba@all{\biba@all\biba@comma\@nameuse{bibali@\biba@one}}%
    }%
    \def\biba@comma{,}%
  }%
  %
  % However, still write in the .aux file a dummy \citation{...} command, so that
  % filters.util.auxfile will still catch those used citations....
  %
  %\immediate\write\@auxout{\noexpand{\bgroup\renewcommand\citation[1]{}\citation{#2}\egroup}}
  \immediate\write\@auxout{\noexpand\bgroup\noexpand\renewcommand\noexpand\citation[1]{}\noexpand\citation{#2}\noexpand\egroup}%
  %
  % Now, produce the \cite command with the original keys instead of the aliases
  %
  \edef\biba@tmp{\the\biba@toks{\biba@all}}%
  \biba@tmp%
}
\makeatother


%
% Now, declare all the alias keys.
%

\bibalias{inspire:10.1103/PhysRev.47.777}{inspire:Phys.Rev.+47+777}
\bibalias{inspire:10.1103/PhysRev.47.777--EPR-paper}{inspire:Phys.Rev.+47+777}
\bibalias{inspire:Hagiwara:2002fs}{inspire:PHRVA+D66+010001}
\bibalias{inspire:PhysRev+47+777}{inspire:Phys.Rev.+47+777}


